\documentclass[twoside,doctor,numberorder]{zjuthesis}
%==============================================================
%==============================================================

%自己需要增加什么 package 或修改什么设置的话,都放在这里吧。
%\usepackage{xxxx}

%==============================================================
%==============================================================
\begin{document}
%==============================================================
%==============================================================
%这部分是论文封面、题名页需要的信息,请根据《研究生学位论文编写规则》自行修改

  % {论文分类号}{论文密级}
  \zjutype{C936}{}
  %论文题目:{中文}{英文}
  \zjutitle{基于成本的钉耙强度与应用——东方视角分析}%
           {Cost Based Rake Strength and Application: An Oriental View}
  %作者:{姓名}{学号}
  \zjuauthor{猪八戒}{10320022}
  %指导教师:{导师}{合作导师}
  \zjumentor{唐三藏}{}
  %个人信息:{研究方向}{专业名称}{学院}
  \zjuinfo{法器工程}{祭坛管理}{大雷音寺管理研究院}
  %日期:{提交日期}{答辩日期中文}{答辩日期英文}
  \zjudate{二〇〇八年九月}{二〇〇八年十二月}{December 2008}
  %论文评阅人:{评阅人中文姓名}{评阅人英文姓名}
  %格式:姓名$\backslash$职称$\backslash$单位
  \zjureviewerone{}{}
  \zjureviewerone{}{}
  \zjureviewerone{}{}
  \zjureviewerone{}{}
  \zjureviewerone{}{}
  %答辩委员会:{委员中文姓名}{委员英文姓名}
  %格式:姓名$\backslash$职称$\backslash$单位
  \zjucommitteemain{弥勒$\backslash$佛$\backslash$大雷音寺}%
                   {Maitreya$\backslash$Buddha$\backslash$Thunder Monastery}
  \zjucommitteeone{观音$\backslash$菩萨$\backslash$补陀落迦山}%
                  {Avalokiteśvara$\backslash$Bodhisattva$\backslash$Potalaka}
  \zjucommitteetwo{地藏王$\backslash$菩萨$\backslash$九华山}%
                  {Ksitigarbha$\backslash$Bodhisattva$\backslash$Jiuhua Mountain}
  \zjucommitteethree{阿氏多$\backslash$罗汉$\backslash$鹫峰山}%
                    {Ajita$\backslash$Arhat$\backslash$Jiufeng Mountain}
  \zjucommitteefour{广成子$\backslash$仙人$\backslash$崆峒山}%
                   {GuangCheng-Zi$\backslash$Immortal$\backslash$Kongtong Mountain}
  \zjucommitteefive{}{}

%==============================================================
% 这部分除了“取舍”外,不需要自己修改,必要信息都已在上面设置。

  %封面
  %%============================================================
%% 中文封面

\thispagestyle{empty}

\vspace{5mm}

\begin{center}
   \includegraphics[width=108mm]{images/zjdx}
\end{center}

\centerline{\songti\erhao\textbf{本科生毕业论文(设计)}}

\vspace{4mm}

\begin{center}
  \includegraphics[width=35mm]{images/standxb}
\end{center}

\vspace{25mm}

{\hspace{16mm}\songti\sanhao\bfseries 题目: \hspace{2mm} \parbox[t]{98mm}{\linespread{1.1}\uline{\zjutitlec}}}

\vspace{7mm}

\begin{tabbing}
    \hspace{30mm} \songti\sihao 姓 \hspace{-2.7mm} \= \songti\sihao 名与学号: \= \underline{\makebox[6cm]{\sihao\zjuauthornamec\hspace{3mm}\zjuauthorid}} \\[2mm]
              \> \songti\sihao 指导教师: \> \underline{\makebox[6cm]{\sihao\zjumentorc}} \\[2mm]
              \hspace{30mm} \songti\sihao 年 \hspace{-2.7mm} \= \songti\sihao 级与专业: \= \underline{\makebox[6cm]{\sihao\zjugrade\hspace{3mm}\zjumajor}} \\[2mm]
              \> \songti\sihao 所在学院: \> \underline{\makebox[6cm]{\sihao\zjucollegec}}
\end{tabbing}


%%============================================================
% empty page for two-page print
\ifthenelse{\equal{\zjuside}{T}}{%
  \newpage\mbox{}%
  \thispagestyle{empty}}{}

%%============================================================
%% English Cover
\newpage
\thispagestyle{empty}

\vspace{5mm}

\begin{center}
    \songti\xiaoyi A Dissertation Submitted to Zhejiang University for the Degree of Bachelor of Engineering
\end{center}

\vspace{4mm}

\begin{center}
  \includegraphics[width=35mm]{images/standxb}
\end{center}

\vspace{25mm}

{\hspace{3mm}\songti\sanhao\bfseries TITLE:\hspace{4mm}\parbox[t]{124mm}{\linespread{1.1}\uline{\zjutitlee}}}

\vspace{7mm}

\begin{tabbing}
    \hspace{18mm} \= \sanhao Author:\hspace{19mm} \= \underline{\makebox[8cm]{\sanhao\zjuauthornamee\hspace{3mm}\zjuauthorid}} \\[2mm]
                  \> \sanhao Supervisor: \> \underline{\makebox[8cm]{\sanhao\zjumentore}} \\[2mm]
                  \> \sanhao Subject: \> \underline{\makebox[8cm]{\sanhao\zjusubject}} \\[2mm]
                  \> \sanhao College: \> \underline{\makebox[8cm]{\sanhao\zjucollegee}} \\[2mm]
                  \> \sanhao Submitted Date: \> \underline{\makebox[8cm]{\sanhao\zjusubmitteddatee}}
\end{tabbing}

%%============================================================
% empty page for two-page print
\ifthenelse{\equal{\zjuside}{T}}{%
  \newpage\mbox{}%
  \thispagestyle{empty}}{}

  %中文题名页
  %%% 中文提名页

\newpage
\thispagestyle{empty}

\vspace{5mm}

\begin{center}
\parbox{0.75\linewidth}{%
  \begin{picture}(0,0)(0,0)
  \setlength{\unitlength}{1cm}
    \put(0,-0.2){\line(1,0){11.3}}
    \put(0,-1.1){\line(1,0){11.3}}
  \end{picture}%
  \linespread{1.1}\xiaoer\bfseries\zjutitlec}
\end{center}

\vspace{5mm}

\begin{center}
  \includegraphics[width=21mm]{images/standxb.pdf}
\end{center}

\vspace{3mm}

\begin{tabbing}
\hspace{30mm} \= \songti\xiaosan\bfseries 论文作者签名: \= \underline{\makebox[5cm]{}} \\[8mm]
              \> \songti\xiaosan\bfseries 指导教师签名: \> \underline{\makebox[5cm]{}}
\end{tabbing}

\vspace{8mm}

\begin{tabbing}
\hspace{10mm} \songti\sihao 论文 \hspace{-2mm} \= \songti\sihao 评阅人 1: \= \underline{\makebox[9cm]{\sihao\zjurevieweronec}} \\
              \> \songti\sihao 评阅人 2: \> \underline{\makebox[9cm]{\sihao\zjureviewertwoc}} \\
              \> \songti\sihao 评阅人 3: \> \underline{\makebox[9cm]{\sihao\zjureviewerthreec}} \\
              \> \songti\sihao 评阅人 4: \> \underline{\makebox[9cm]{\sihao\zjureviewerfourc}} \\
              \> \songti\sihao 评阅人 5: \> \underline{\makebox[9cm]{\sihao\zjureviewerfivec}}
\end{tabbing}

\vspace{8mm}

\begin{tabbing}
\hspace{5mm}\songti\sihao 答辩委员 \hspace{-2.2mm} \= \songti\sihao 会主席: \= \underline{\makebox[9cm]{\sihao\zjucommitteemainc}} \\
          \>    \songti\sihao ~委员 1: \> \underline{\makebox[9cm]{\sihao\zjucommitteeonec}} \\
          \>    \songti\sihao ~委员 2: \> \underline{\makebox[9cm]{\sihao\zjucommitteetwoc}} \\
          \>    \songti\sihao ~委员 3: \> \underline{\makebox[9cm]{\sihao\zjucommitteethreec}} \\
          \>    \songti\sihao ~委员 4: \> \underline{\makebox[9cm]{\sihao\zjucommitteefourc}} \\
          \>    \songti\sihao ~委员 5: \> \underline{\makebox[9cm]{\sihao\zjucommitteefivec}}
\end{tabbing}

\vspace{8mm}

\begin{tabbing}
\hspace{34mm} \= \songti\sihao 答辩日期: \= \underline{\makebox[5cm]{\songti\sihao\zjudefencedatec}} \\
\end{tabbing}

\ifthenelse{\equal{\zjuside}{T}}{%
  \newpage\mbox{}%
  \thispagestyle{empty}}{}

  %英文题名页
  %%% 英文提名页

\newpage
\thispagestyle{empty}

\vspace{5mm}

\begin{center}
\parbox{0.75\linewidth}{%
  \begin{picture}(0,0)(0,0)
  \setlength{\unitlength}{1cm}
    \put(0,-0.2){\line(1,0){11.3}}
    \put(0,-1.1){\line(1,0){11.3}}
  \end{picture}%
  \linespread{1.4}\large\bfseries\zjutitlee}
\end{center}

\vspace{5mm}

\begin{center}
  \includegraphics[width=21mm]{images/standxb.pdf}
\end{center}

\vspace{-6mm}

\begin{tabbing}
\hspace{20mm} \= \large\bfseries Supervisor's signature: \= \underline{\makebox[5cm]{}}\kill \\
              \> \hspace{7mm} \large\bfseries Author's signature: \> \underline{\makebox[5cm]{}} \\[5mm]
              \> \large\bfseries Supervisor's signature: \> \underline{\makebox[5cm]{}}
\end{tabbing}

\vspace{8mm}

\begin{tabbing}
\large External Reviewers: \= \underline{\makebox[10.8cm]{\zjurevieweronee}} \\
                    \> \underline{\makebox[10.8cm]{\zjureviewertwoe}} \\
                    \> \underline{\makebox[10.8cm]{\zjureviewerthreee}} \\
                    \> \underline{\makebox[10.8cm]{\zjureviewerfoure}} \\
                    \> \underline{\makebox[10.8cm]{\zjureviewerfivee}} \\[8mm]
\large Examining Committe Chairperson: \= \\
\hspace{41mm} \= \underline{\makebox[10.8cm]{\zjucommitteemaine}} \\
\large Examining Committe Members: \= \\
\hspace{41mm} \= \underline{\makebox[10.8cm]{\zjucommitteeonee}} \\
             \> \underline{\makebox[10.8cm]{\zjucommitteetwoe}} \\
             \> \underline{\makebox[10.8cm]{\zjucommitteethreee}} \\
             \> \underline{\makebox[10.8cm]{\zjucommitteefoure}} \\
             \> \underline{\makebox[10.8cm]{\zjucommitteefivee}}
\end{tabbing}

\vspace{6mm}

\begin{tabbing}
\hspace{28mm} \= \large Date of oral defence: \underline{\makebox[5cm]{\zjudefencedatee}}
\end{tabbing}

\ifthenelse{\equal{\zjuside}{T}}{%
  \newpage\mbox{}%
  \thispagestyle{empty}}{}
 % 硕士论文请根据需要取舍。
  %独创性声明
  %% 诚信承诺书

\newpage
\thispagestyle{empty}

\begin{center}
\heiti\xiaosan 浙江大学本科生毕业论文(设计)诚信承诺书
\end{center}

\vspace{5mm}

{\songti\sihao
\begin{enumerate}
  \item 本人郑重地承诺所呈交的毕业论文(设计),是在指导教师的指导下严格按照学校和学院有关规定完成的。
  \item 本人在毕业论文(设计)中引用他人的观点和参考资料均加以注释和说明。
  \item 本人承诺在毕业论文(设计)选题和研究内容过程中没有抄袭他人研究成果和伪造相关数据等行为。
  \item 在毕业论文(设计)中对侵犯任何方面知识产权的行为,由本人承担相应的法律责任。
\end{enumerate}

\vspace{12mm}
\hspace{20mm}毕业论文(设计)作者签名:
\begin{flushright}
    \underline{\hspace{4em}} 年 \underline{\hspace{2em}} 月 \underline{\hspace{2em}} 日
\end{flushright}}

\ifthenelse{\equal{\zjuside}{T}}{%
  \newpage\mbox{}%
  \thispagestyle{empty}}{}


  \frontmatter   
  \pagenumbering{Roman}

  %勘误页
  %%% 勘误页
\chapter*{\centerline{勘\quad误}}
\chaptermark{勘误}
\addcontentsline{toc}{chapter}{勘误}
  % 请根据需要取舍。
  %致谢
  %致谢
\chapter*{\centerline{致\quad 谢}}
\chaptermark{致谢}
\addcontentsline{toc}{chapter}{致谢}

\vspace{2em}


  %序言
  %%% 序言
\chapter*{\centerline{序\quad言}}
\chaptermark{序言}
\addcontentsline{toc}{chapter}{序言}
 % 请根据需要取舍。
  %中文摘要
  %% 中文摘要
\chapter*{\centerline{摘\quad 要}}
\chaptermark{摘要}
\addcontentsline{toc}{chapter}{摘要}

\vspace{1em}
大多数机器学习的算法都可以看作是寻找一个函数或模型$f$让它的经验误差达到最小。
但是这只有在你有足够多的训练数据的时候才能工作得很好。现在人们逐渐开始关注训练
数据稀少时的学习问题,正则化就显得越来越重要了。通过在原始的目标函数中加入一个
额外的正则化项,我们可以对解空间进行限制,从而实现增加算法的泛化性能。不过,现在
的大多数正则化的技术都没有专门考虑时间正则化的问题。考虑到现在有大量的会随着时间
而逐步发展的数据(例如动态的web页面、博客内容以及股票价格等),我们提出一个新
的正则化方法,同时考虑数据在时间和空间上的平滑性质。我们把这个新的正则化方法应用
到视频压缩里,实验结果展示了这种算法的效力。

\vspace{1em}

\noindent\textbf{关键词}:\quad 机器学习,正则化,视频压缩

  %英文摘要
  %% 英文摘要
\chapter*{\centerline{Abstract}}
%\chaptermark{Abstract}
%\addcontentsline{toc}{chapter}{Abstract}

\vspace{1em}

Rake is good. My rake has nine prongs. Rake can be used in farm. Blur, blur, blur...

\vspace{1em}

\noindent\textbf{Keywords:~~rake,~~cost,~~supernatural utensil}


%==============================================================
%这部分不需要自己修改。

  %目次页
  \tableofcontents
  \addtocontents{toc}{\protect\chaptermark{目次}}
  \addtocontents{toc}{\protect\contentsline {chapter}{\protect\makebox[\linewidth]{目次\hfill}\vspace{-2em}}{}}
  %插图和附表清单
  \listoffigures
  \addtocontents{lof}{\protect\chaptermark{图目录}}
  \listoftables
  \addtocontents{lot}{\protect\chaptermark{表目录}}
  %术语表
  \printnomenclature
  \chaptermark{术语表}

  \mainmatter

%==============================================================

  \chapter{绪论}

\section{机器学习与正则化}

机器学习是人工智能的一个子领域,它着眼于设计和开发能自动从数据中生成出模型的算法。机器学习又
被细分为监督学习、无监督学习以及半监督学习、增强学习等类别,但是它们的思路都是相似的,亦即对
问题进行建模,并寻找一个最适合给定约束条件的函数或模型,通常约束条件直接来自于训练数据。如果
训练数据的数量足够多,我们通常能得到比较好的结果,然而大多数时候训练数据总是不够的,此时我们
通常会遇到有很多(甚至无穷多个)满足约束条件的解的情况,并且大部分这样的解在未知数据上的表现
非常差。换句话说,这样的问题是数值不稳定的,或者说泛化性能很差。

一种解决办法就是正则化(Regularization),亦即对可能解空间进行限制,这种技术最初由
Tikhonov 和 Arsenin \cite{tikhomirov1960dsf} 提出,用于解决矩阵求逆的问题,并且后
来被成功地应用到机器学习中来。许多机器学习的算法(例如支持向量机)都可以看成是某
种形式的正则化。简单来说,正则化可以看作是关于待解决问题的某种先验的知识,例如,在
脊回归(Ridge Reguression)中的正则化可以看作是将参数的先验分布定为高斯分布的结果,
而流形正则化(Manifold Regularization)\cite{On-Manifold-Regularization} 则是编码了
数据点都分布在一个低维流形上这样一个先验假设。

虽然正则化如今已经成为机器学习中广泛使用的一种技术,大多数情况人们都只考虑了问题在空间上的
属性。当我们遇到随时间变化的数据(例如,动态网页、博客内容或者股票价格等)时,一个很自然的
想法就是要保证数据在时间上的平滑性,这样的假设通常能帮助我们得到更健壮的解。

\section{写点什么吧}

恩,反正这一章就是绪论。

  \chapter{相关的工作}

在这一章中,我们简要地介绍 Cheng 的视频压缩方法
\cite{learning-to-compress-images}
,这是和我们的方法最相关的工作。从机器学习的视觉来看,视频着色可以被理解为一个半监督学习
的过程。给定一个视频帧的一些有颜色的像素点(有标签的数据)和一些灰度像素点(无标签的数据),我们
希望学习一个模型,用于在灰度信息的基础上预测当前以及后续一些帧的颜色值(标签)。

接下来我们将介绍用于半监督学习的拉普拉斯正则化的最小二乘法(LapRLS)。使用
LapRLS
来进行着色是基于这样的一个假设:如果两个点有类似的灰度值并且在空间上比较接近,那么他们的颜色值也很
有可能是非常相似的。我们用$\textbf{z} \in \mathbb{R}^d$
来表示有标签的数据点,$\textbf{x} \in \mathbb{R}^d$
来表示任意数据点(有标签或者没有标签的)。

考虑这样一个线性回归模型$y=\textbf{a}^T \textbf{x} +
\epsilon$,其中$y$是因变量,$\textbf{x}$是自变量,$\textbf{a}$是权重向量,而$\epsilon$是一个
位置的期望为零的误差。不同的观察值有不同的误差,这些误差之间相互独立,但是方差都等于$\sigma^2$。在
给定输入向量$\textbf{x}$和权重向量$\textbf{a}$的前提下,我们定义
$f(\textbf{x})=\textbf{a}^T\textbf{x}$为模型的输出。

LapRLS算法同时使用有标签的数据和无标签的数据来学习回归模型$f$。假设一共有$m$个点,其中$k$个有
标签。令$S$为相似度矩阵,LapRLS算法对如下优化问题进行求解:

\begin{eqnarray}
&J_{LapRLS}(\textbf{a}) =\sum_{i=1}^k \big( f(\textbf{z}_i) - y_i \big)^2 \nonumber \\
& + \frac{\lambda_1}{2} \sum_{i,j=1}^{m} \big( f(\textbf{x}_i) -
f(\textbf{x}_j)\big)^2 S_{ij} + \lambda_2 \|\textbf{a}\|^2
\label{eqn:LPP-least-square-error}
\end{eqnarray}

其中$y_i$是数据点$z_i$的标签。在我们选取的对称权重$S_{ij}$($S_{ij}=S_{ji}$)下,给定的损失函数
将不允许相邻的点$\textbf{x}_i$和$\textbf{x}_j$被影射到互相远离的地方。因此,最小化
$J_{LapRLS}(\textbf{a})$就可以保证如果$\textbf{x}_i$和$\textbf{x}_j$相互接近的话,$f(\textbf{x}_i)$
和$f(\textbf{x}_j)$之间也会比较接近。对于相似度矩阵$S$有许多不同的选择,一个简单的定义如下所示:

\begin{equation}
S_{ij}=\left\{%
\begin{array}{ll}
    1, &
    \hbox{如果$\textbf{x}_i$在$\textbf{x}_j$的$p$个最接近的邻居中,}\\
    & \hbox{或者$\textbf{x}_j$在$\textbf{x}_i$的$p$个最接近的邻居中;} \\
    0, & \hbox{其他情况.} \\
\end{array}%
\right. \label{eqn:similarity}
\end{equation}

令$D$为一个对角阵,$D_{ii}=\sum_j
S_{ij}$并且$L=D-S$。矩阵$L$在谱图理论中被称为拉普拉斯算子。令$Z=(\textbf{z}_1,
\cdots, \textbf{z}_k)$, $X=(\textbf{x}_1, \cdots,
\textbf{x}_m)$,$\textbf{y}=(y_1, \cdots,
y_k)^T$,则最小化问题\ref{eqn:LPP-least-square-error}的解由下式给出:

\begin{equation}\label{eqn:LapRLS-solution}
\widehat{\textbf{a}}=\big(ZZ^T + \lambda_1 XLX^T +\lambda_2 I
\big)^{-1} Z\textbf{y}
\end{equation}

其中$I$是一个$d \times
d$的单位矩阵。LapRLS也可以在再生核希尔伯特空间(Reproducing Kernel
Hilbert Space,
RKHS)中进行,这将到处一个非线性的解。关于LapRLS的更多细节请参见\cite{Manifold-Regularization-Journal}。



%==============================================================
%这也是个不需要自己修改的部分。

  \backmatter %结束章节自动编号

  %参考文献
  \addcontentsline{toc}{chapter}{参考文献} % 解决目录中没有相应的参考文献的条目问题
  \chaptermark{参考文献}

%==============================================================

  \bibliography{bibfile/zjubib}

  %附录
  \chapter*{附\quad 录}
\chaptermark{附录}
\addcontentsline{toc}{chapter}{附录} 


  % 这里应该还有个 indices,但我从没见过有使用 indices 的论文,略之。
  %作者简历
  \chapter*{攻读博士学位期间主要研究成果}
\chaptermark{攻读博士学位期间主要研究成果}
\addcontentsline{toc}{chapter}{攻读博士学位期间主要研究成果} 

猪八戒,又名猪刚鬣和天蓬元帅,法号“悟能”,猪脸人身,九齿钉耙为其武器。

猪八戒原是天庭中统领十万天河水兵的天蓬元帅,由于蟠桃会上喝酒醉后调戏月宫仙女嫦娥,打了两千锤后被贬下凡,又投错胎变成猪模样,在高老庄抢占民女。后经观音菩萨指点,拜唐僧为师,一同赴西天取经。取回真经后,猪八戒由于“又有顽心,色情未泯”被封为净坛使者。

后人有鉴于天蓬元帅颇具凡心,善於戏弄逗笑,极具娱乐效果,遂奉为娱乐界、青楼祖师爷。

\begin{itemize}
\item 发表的文章
\item 出版的书籍
\end{itemize}

%==============================================================
%==============================================================
\end{document}
