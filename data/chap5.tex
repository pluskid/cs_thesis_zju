\chapter{结合主动学习和半监督学习的视频压缩}

在本章中我们将介绍如何结合主动学习核半监督学习来进行视频压缩。我们使
用PSNR来衡量图像压缩的质量。

同之前的视频着色和压
缩\cite{learning-to-compress-images,colorization-using-optimization}一
样,我们将使用$YUV$空间。其中$Y$是灰度通道,而$U$和$V$则是存储颜色的色
度通道。我们将独立地预测$U$和$V$两个通道的值。
在\cite{learning-to-compress-images} 中,用于表示一个像素点的特征包括空
间信息和局部的纹理信息。我们经过实验发现局部纹理信息对于预测颜色值来说
并没有什么帮助,因此,我们仅使用空间信息和灰度值来表达每个像素点的特
征。

对于一个一共有$\ell$帧,每一帧有$n$个像素点的视频。首先,我们在第一帧上
应用第 \ref{chap:semi-supervised-learning} 章中所描述的半监督学习算法来
学习一个模型,然后用这个模型来预测该帧以及后续一些帧的颜色值。当PSNR值
低于一定阈值时,我们会重新训练一个模型。半监督学习的过程需要对一一个$m
\times m$的稠密矩阵求逆,这是非常大的计算量。为了减轻计算负担,Chen et
al. \cite{learning-to-compress-images} 建议使用 NCut
\cite{learning-a-classification-model-for-segmentation} 将图像分割成一
些小区域,然后使用 {\em super-pixel} 来表示每个小区域,并在这些小区域
中随机地选择像素点来参与计算。不过 NCut 本身就是一个很耗时的算法。在我
们的工作中,我们采用直接将图像划分为小方格个形式,从而避免了复杂的分割
算法。在我们实验的实验中,我们将小方格的个数定位 2000 个。然后我们会随
机地从每个小方格选取一个像素点,这样总共会有 2000 个像素点,然后我们会
在这些像素点上构造 4-邻接图。

一旦构造好了邻接图,我们就可以应用主动学习的算法来选取最具有代表性的像
素点。视频压缩的质量会随着所选的像素点的数量变化而变化,选的点越多,质
量就会越高。另一方面,选择更多的点会降低压缩比。由于图是在 2000 个点上
构造的,因此最多可以选择 2000 个点。解码的过程就是应用半监督学习算法来
学习一个模型,并用它来预测灰度像素点的颜色值。

\begin{figure*}[t]
  \center \subfigure[Original
  Frames]{\includegraphics[width=.4\linewidth]{images/telemarket-frames}}
  \hspace{4mm} \subfigure[Colorized
  Frames]{\includegraphics[width=.4\linewidth]{images/telemarket-rcv}}
\caption{The first 48 frames of video
  telemarket. We put a yellow film icon where our method trained a new
  model and a red film icon for Cheng's Approach.}
\label{fig:telemarket}
\end{figure*}


\begin{table*}[t]
  \centering \scriptsize
  \caption{Video compression comparison. cr denotes compression ratio.}
  \label{tab:more-example}
  \begin{tabular}{|l|c|c|c|c|c|c|c|c|c|c|}
    \hline
    video & orig. size & gray. size & \multicolumn{2}{|c|}{Cheng's approach} & \multicolumn{2}{|c|}{SFM} & \multicolumn{2}{|c|}{MFM} & \multicolumn{2}{|c|}{MPEG-4} \\
    \cline{4-11}
    & & & cr & PSNR & cr & PSNR & cr & PSNR & cr & PSNR \\
    \hline
    claire & 463452 & 44556 & 10.48\% & 29.7 & 10.48\% & 35.2 & 10.65\% & \textbf{35.6} & 10.48\% & 35.0 \\
    \hline
    miss-america & 229148 & 17082 & 11.82\% & 35.7 & 11.82\% & 36.4 & 10.60\% & \textbf{37.0} & 11.09\% & 36.0 \\
    \hline
    akiyo & 226780 & 26588 & 12.61\% & 28.3 & 12.61\% & 33.0 & 13.31\% & 33.6 & 12.77\% & \textbf{33.9} \\
    \hline
    suzie & 466278 & 98926 & 34.08\% & 38.5 & 33.23\% & 40.3 & 30.61\% & \textbf{41.1} & 39.29\% & 40.7 \\
    \hline
    mother-daughter & 466836 & 117302 & 28.98\% & 38.2 & 28.55\% & 40.4 & 27.70\% & \textbf{41.2} & 27.54\% & 39.9 \\
    \hline
  \end{tabular}
\end{table*}

